\chapter*{Zusammenfassung}
Go ist eine Programmiersprache, welche einen besonderen Fokus auf 
die Erzeugung von effizienten nebenläufigen Programmen legt. Dazu 
stellt Go s.g. Go-Routinen als leichtgewichtige Threads sowie 
Mutexe und Channels zur Synchronisation von und Kommunikation zwischen 
solchen Routinen bereit. Solche Konstrukte können aber leicht zu 
Concurrency-Bugs führen, welche auf Grund ihre starken Abhängigkeit von der 
genauen Ausführung des Programms nur schwer erkennbar sind.\\
Diese Arbeit entwickelt und implementiert einen dynamischen Detektor zur
Erkennung solcher Probleme. Der Detektor verbindet dabei die Detektion 
von Bugs durch Mutexe mit Lock-Bäumen, sowie die Betrachtung von 
Bugs auf Channels, sowie deren potenzielle Kommunikationspartner mit Hilfe von 
Vector-Clocks. Außerdem ist der Detektor in der 
Lage mehrere, durch Select-Statement verzweigte Ausführungspfade zu betrachten.