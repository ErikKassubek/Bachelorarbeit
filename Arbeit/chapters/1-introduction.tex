\chapter{Einführung}\label{chap:introduction}
Go ist eine seit 2007 von Google entwickelte Programmiersprache, welche sich 
in den letzten Jahren zu einer der beliebtesten Programmiersprachen entwickelt 
hat~\cite{ranking}. Ein Fokus der Sprache liegt dabei auf der Erstellung
sicherer und effizienter nebenläufiger Programme. Go stellt verschiedene 
Funktionen für die Erstellung solcher nebenläufiger Programme zur Verfügung.
Die wichtigsten sind dieser Funktionen sind dabei Go-Routinen, Mutexe und 
Channels.\\Go-Routinen erlauben das nebenläufige Ausführen mehrerer Codeteile.
Mutexe erlauben die Synchronisation von Routinen, sowie die Lösung des Problems 
des kritischen Ausschlusses. 
Channel können ebenfalls für die Synchronisation verwendet werden,
können aber zusätzlich für eine Kommunikation, bzw.\ das Senden von Daten zwischen 
Routinen verwendet werden. Programmierer werden dabei angehalten, Channel
anstelle von gemeinsamen Variablen zu verwenden, da bei diesen die Wahrscheinlichkeit
für das Auftreten von Concurrency-Bugs geringer ist~\cite{sharedMemory}.\\
Solche Bugs (z.B. Deadlocks), welche durch die Nebenläufigkeit von Programmen sowie ihrer 
Synchronisationsmechanismen entstehen sind in der Praxis aber dennoch sehr weit 
verbreitet~\cite{numberBugs}.

Diese Arbeit beschäftigt sich mit der Entwicklung und Implementierung
eines dynamischen Detektors ``GoChan'' für die automatische Erkennung
von Problemen, welche durch die Verwendung von Mutexen und Channels 
in nebenläufigen Go-Programmen entstehen. Der größte Fokus wird dabei 
auf die Erkennung und Analyse von Deadlock-Situationen, sowie das vorzeitige 
schließen von Channels gelegt, da diese zu einem Absturz eines Programms 
führen können. Der Detektor vereinigt dabei verschiedene Methoden aus 
iGoodLock~\cite{iGoodLock} und UNDEAD~\cite{Undead} sowie 
Gopherlyzer-GoScout~\cite{PPDP18} und GFuzz~\cite{gfuzz} zur 
Erkennung und Analyse von 
durch Mutenxen und Channels erzeugten Bugs.\\
GoChan basiert dabei aus zwei Programmen, einem Instrumenter und einem 
Analyzer, für welche jeweils eine 
Implementierung vorliegt. Für die Analyze wird das zu analysierende Programm 
mehrfach durchlaufen. Bei jedem Durchlauf zeichnet der Analyzer relevante 
Situationen auf und führt anschließend, basieren auf dem so erstellten Trace 
eine Analyse aus, um Probleme zu erkennen. Die Analyse wird mehrfach wiederholt, 
um verschiedene Ausführungspfade zu analysiere, welche durch die Verwendung 
verschiedener Pfade in Select-Operationen erzeugt werden. Um den Trace eines 
Programmdurchlaufs aufzeichnen zu können und bei Select-Operationen verschiedene 
Ausführungspfade zu erzwingen, muss der zu analysierende Programmcode 
verändert werden. Dies kann durch den Instrumenter automatisiert erfolgen.

Der Aufbau dieser Arbeit ist wie folgt. Zuerst wird in Abschnitt~\ref{chap:background}
die theoretische Funktionsweise des Detektors beschrieben und dargelegt, 
welche Art von Situationen der Detektor erkennen soll. 
Abschnitt~\ref{Chap:Instrumenter} geht auf die Funktionsweise und Implementierung 
des Instrumenters und Abschnitt~\ref{Chap:Implement} auf die Implementierung 
des Analyzers ein. In Abschnitt~\ref{Chap:Eval} wird der Detektor auf 
konstruierte und tatsächliche Programme angewandt, um zu überprüfen, 
wie gut er in der Lage ist, Probleme richtig einzusätzen. Für die 
tatsächlichen Programme werden dabei Programme aus GoBench~\cite{gobench}
verwendet, welche eine Sammlung von Programmen mit Concurrency-Bugs aus 
mehreren großen open-source Programmen besitzt. Zuletzt werden die 
Erkenntnisse in Abschnitt~\ref{chap:conclusion} zusammengefasst. Die Implementierung
der beschriebenen Programme befindet sich in \url{https://github.com/ErikKassubek/GoChan}.
