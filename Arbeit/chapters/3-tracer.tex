\chapter{Tracer}\label{Chap:Tracer}
Um ein Program analysieren zu können, wird der Ablauf eines Programmdurchlaufs
aufgezeichnet. Dazu wurde ein Tracer implementiert, durch welchem die 
Channel- und Lock-Operationen, sowie andere Operationen wie Select und das 
erzeugen einer neuen Routine, ersetzt, bzw. erweitert werden. Diese 
führen die eigentlichen Operationen aus und zeichnen gleichzeitig den 
Trace auf. Die Ersetzung durch die Drop-In Replacements kann dabei automatisch 
durch einen Instrumenter erfolgen, welche mit Hilfe des Abstract Syntax Tress 
die Ersetzungen vornimmt.\\
Anders als in vielen anderen Programmen, welche den Trace von Go-Programmen
analysieren, wie z.B.~\cite{GoAt2} oder~\cite{GoVis} wird dabei der Tracer 
selbst implementiert und basiert nicht auf dem Go-Runtime-Tracer~\cite{GoRunTrace}. 
Dies ermöglicht es, den Tracer genau auf die benötigten Informationen zuzuschneiden
und so einen geringeren negativen Einfluss auf die Laufzeit des Programms zu erreichen.

\extend{Tracer Einführung}

\section{Trace}\label{Chap:Tracer-Sec:Trace}
\draft{Trace} 
Der Aufbau des Trace basiert auf~\cite{PPDP18}. Er wird aber um Informationen 
über Locks erweitert. Der Trace wird für jede Routine
separat aufgezeichnet. Außerdem wird, anders als in~\cite{PPDP18} ein globaler
Program-Counter für alle Routinen und nicht ein separater Counter für jede 
Routine verwendet. Dies ermöglicht es bessere Rückschlüsse über den genauen 
Ablauf des Programms zu ziehen.  
Die Syntax des Traces in EBNF gibt sich 
folgendermaßen:
\begin{align*}
  \begin{matrix*}[l]
    T & = & ''\ [\ '',\ \{\ U\ \},\ ''\ ]\ ''; & \text{Trace}\\
    U & = & ''\ [\ '',\ \{\ t\ \},\ ''\ ]\ ''; & \text{lokaler Trace} \\
    t & = & signal(t, i)\ |\ wait(t, i)\ |\ pre(t, as)\ |\ post(t, i, x!) & \text{Event}\\
      &   & |\ post(t, i, x?, t') |\ post(t, default)\ |\ close(t, x)\  & \\
      &   & |\ lock(t, y, b, c) |\ unlock(t, y); & \\
    a & = & x,\ (''\ !\ ''\ |\ ''\ ?\ ''); & \\
    as & = & a\ |\ (\{\ a\ \}, [\ ''\ default\ ''\ ]); & \\
    b & = & ''-''\ |\ ''t''\ |\ ''r''\ |\ ''tr'' & \\
    c & = & ''0''\ |\ ''1''
  \end{matrix*}
\end{align*}
wobei $i$ die Id einer Routine, $t$ einen globalen Zeitstempel, $x$ die Id eines 
Channels und $y$ die Id eines Locks darstellt. Die Events haben dabei folgende Bedeutung:
\begin{itemize}
  \item \texttt{signal(t, i)}: In der momentanen Routine wurde
    eine Fork-Operation ausgeführt,
    d.h. eine neue Routine mit Id $i$ wurde erzeugt.
  \item \texttt{wait(t, i)}: Die momentane Routine mit Id $i$ wurde soeben erzeugt. Dies ist 
    in allen Routinen außer der Main-Routine das erste Event in ihrem lokalen Trace.
  \item \texttt{pre(t, as)}: Die Routine ist an einer Send- oder Receive-Operation eines 
    Channels oder an einem Select-Statement angekommen, dieses wurde aber noch nicht 
    ausgeführt. Das Argument $as$ gibt dabei die Richtung und den Channel an. 
    Ist $as = x!$, dann befindet sich der Trace vor einer Send-Operation, bei 
    $as = x?$ vor einer Receive-Operation. Bei einem Select-Statement ist 
    $as$ eine Liste aller Channels für die es einen 
    Case in dem Statement gibt. Besitzt das Statement einen Default-Case, wird
    dieser ebenfalls in diese List aufgenommen.
  \item \texttt{post(t, i, x!)}: Dieses Event wird in dem Trace gespeichert, nachdem 
    eine Send-Operation auf $x$ erfolgreich abgeschlossen wurde. 
    $i$ gibt dabei die Id der sendenden Routine
    an.
  \item \texttt{post(t, i, x?, t')}: Dieses Event wird in dem Trace gespeichert, nachdem 
    eine Receive-Operation des Channels $x$ erfolgreich abgeschlossen wurde. 
    $i$ gibt dabei die Id der sendenden Routine an. $t'$ gibt den Zeitstempel an,
    welcher bei dem Pre-Event der sendenden Routine galt. Durch die Speicherung der Id und des 
    Zeitstempels der sendenden Routine bei einer Receive-Operation lassen sich 
    die Send- und Receive-Operationen eindeutig zueinander zuordnen.
  \item \texttt{post(t, default)}: Wird in einem Select-Statement der Default-Case ausgeführt,
    wird dies in dem Trace der entsprechenden Routine durch $post(t, default)$ 
    gespeichert.
  \item \texttt{close(t, x)}: Mit diesem Eintrag wird das schließen eines Channels $x$ 
    in dem Trace aufgezeichnet.
  \item \texttt{lock(t, y, b, c)}: Der Beanspruchungsversuch eines Locks mit id $y$ wurde beendet. 
    $b$ gibt dabei die Art der Beanspruchung an. Bei $b = r$ war es eine R-Lock
    Operation, bei $b = t$ eine Try-Lock Operation und bei $b = tr$ ein Try-R-Lock
    Operation. Bei einer normalen Lock-Operation ist $b = -$. Bei einer 
    Try-Lock Operation kann es passieren, dass die Operation beendet wird, 
    ohne das das Lock gehalten wird. In diesem Fall wird $c$ auf $0$, und 
    sonst auf $1$ gesetzt. 
  \item \texttt{unlock(t, y)}: Das Lock mit id $y$ wurde zum Zeitpunkt 
    $t$ wieder freigegeben. 
\end{itemize}
Um diesen Trace zu erzeugen, werden die Standartoperation aud Go durch Elemente
des Tracers ersetzt. Die Funktionsweisen dieser Ersetzungen sind im folgenden 
angegeben. Dabei werden nur solche Ersetzungen angegeben, welche direkt 
für die Erzeugung
des Traces notwendig sind. Zusätzlich werden noch 
weitere Ersetzungen durchgeführt, wie z.B. die Ersetzung der Erzeugung von 
Mutexen und Channel von den Standardvarianten zu den Varianten des Tracers.
Hierbei wird auch die Größe jedes Channels gespeichert.
Dies werden in der Übersicht zur Vereinfachung nicht betrachte. Auch werden 
in der Übersicht nur die Elemente betrachtet, die für die Durchführung der 
Operation und dem Aufbau des Traces benötigt werden. Hilfselemente, wie z.B. 
Mutexe, welche verhindern, dass mehrere Routinen gleichzeitig auf die selbe 
Datenstruktur, 
z.B. die Liste der Listen, welche die Traces für die einzelnen Routinen 
speichern, zugreifen, werden nicht mit angegeben. Dabei sei $c$ ein 
Zähler, $nR$ ein Zähler für die Anzahl der Routinen, $nM$ ein Zähler für die 
Anzahl der Mutexe und $nC$ ein Zähler für die Anzahl der Channels. $nM$ und $nC$
werden bei der Erzeugung eines neuen Mutex bzw. eines neuen Channels atomarisch 
Incrementiert. Den erzeugten Elementen wird er neue Wert als $id$ zugeordnet. All diese 
Zähler seien global und zu Beginn als $0$ initialisiert. Außerdem bezeichnet 
$mu$ einen Mutex, $rmu$ einen RW-Mutex, $ch$ einen Channel und $B$ bzw. $B_i$
mit $i\in\mathbb{N}$ den 
Körper einer Operation. Zusätzlich
sei $id$ die $Id$ der Routine, in der eine Operation ausgeführt wird,
$[signal(t, i)]^{id}$ bedeute, dass der das entsprechende Element (hier als 
Beispiel $signal(t, i))$, in den Trace der Routine mit id $id$ eingeführt wird
und $[+]^i$ bedeute, das in die Liste der Traces ein neuer, leerer Trace 
eingefügt wird, welcher für die Speicherung des Traces der Routine $i$ 
verwendet wird. 
$\langle a|b\rangle$ bedeutet, dass ein Wert je nach Situation auf $a$ oder $b$ gesetzt 
wird. Welcher Wert dabei verwendet wird, ist aus der obigen Beschreibung der 
Trace-Elemente erkennbar. $\text{e}_1$ bis $\text{e}_n$ bezeichnet die Selektoren in einem Select statement.
$\text{e}_i^*$ bezeichnet dabei einen Identifier für einen Selektor, der sowohl die 
Id des beteiligten Channels beinhaltet, als auch die Information, ob es sich um ein 
Send oder Receive handelt und $\text{e}_i^m$ die Message, die in einem Case 
empfangen wurde. 
\begin{tabular}{lcl}
  go B & $\Rightarrow$ & nr := atomicInc(nR); ts := atomicInc(c); [ signal(ts, nr) ]$^\text{nr}$;\\
    & & [+]$^\text{nr}$; go \{ ts' := atomicInc(c); [ wait(ts, nr) ]$^\text{id}$; B\};\\
  ch <- i & $\Rightarrow$ & ts := atomicInc(c); [ pre(ts, ch.id, true) ]$^\text{id}$; ch <- \{i, id, ts\};\\
    & & ts' := atomicInc(c); [ post(ts', ch.id, true, id) ]$^\text{id}$\\
  <- ch & $\Rightarrow$ & ts := atomicInc(c); [ pre(ts, ch.id, false) ]$^\text{id}$;\\
    & & \{i, id\_send, ts\_send\} := <-c; ts' := atomicInc(c);\\
    & & [ post(ts', ch.id, false, id\_send, ts\_send) ]$^\text{id}$; return i;\\
  close(ch) & $\Rightarrow$ & ts := atomicInc(c); close(ch); [ close(ts, ch.id) ]$^\text{id}$\\
  select(e$_\text{i} \leadsto \text{B}_\text{i}$) & $\Rightarrow$ & ts := atomicInc(c); [ pre(ts, e$_1^*$, $\ldots$, e$_n^*$, false) ]$^\text{id}$;\\
    & & select(e$_\text{i} \leadsto$ \{ ts' := atomicInc(c);\\
    & & [ $\langle \text{post(ts, e$_\text{i}$.ch, false, e$_\text{i}^\text{m}$.id\_send, e$_\text{i}^\text{m}$.ts\_send)}\ |$ \\
    & & post(ts, e$_\text{i}$.ch, true, id) $\rangle$ ]$^\text{id}$ B$_\text{i}$\}) \\
  select(e$_\text{i} \leadsto \text{B}_\text{i}$ | B$_\text{def}$) & $\Rightarrow$ & ts := atomicInc(c); [ pre(ts, e$_1^*$, $\ldots$, e$_n^*$, false) ]$^\text{id}$;\\
    & & select(e$_\text{i} \leadsto$ \{ ts' := atomicInc(c);\\
    & & [ $\langle \text{post(ts, e$_\text{i}$.ch, false, e$_\text{i}^\text{m}$.id\_send, e$_\text{i}^\text{m}$.ts\_send)}\ |$ \\
    & & post(ts, e$_\text{i}$.ch, true, id) $\rangle$ ]$^\text{id}$ B$_\text{i}$\} |\\
    & & ts' := atomicInc(c); [ default(ts) ]$^\text{id}$; B$_\text{def}$) \\
  mu.(Try)Lock() & $\Rightarrow$ & ts := atomicInc(c); mu.(Try)Lock();\\
    & & [ lock(ts, mu.id, $\langle \text{-|t}\rangle$, $\langle \text{0|1}\rangle$) ]$^\text{id}$;\\
  mu.Unlock() & $\Rightarrow$ & ts := atomicInc(c); mu.Unlock(); [ unlock(ts, mu.id) ]$^\text{id}$;\\
  rmu.(Try)(R)Lock() & $\Rightarrow$ & ts := atomicInc(c); rmu.(Try)(R)Lock();\\
    & & [ lock(ts, rmu.id, $\langle \text{-|t|r|tr}\rangle$, $\langle \text{0|1}\rangle$) ]$^\text{id}$;\\
  rmu.Unlock() & $\Rightarrow$ & ts := atomicInc(c); rmu.Unlock(); [ unlock(ts, rmu.id) ]$^\text{id}$;
\end{tabular}

Man betrachte als Beispiel das folgende Programm in Go:
\begin{figure}figure
  \lstinputlisting{code/03-tracer/example_code_pre.txt}
  \caption{Beispielprogramm für Tracer}
  \label{Chap:Tracer-Sec:Trace-Fig:Example}
\end{figure}
% Erweitert man diesen mit dem Tracer, erhält man folgendes Programm:
% \lstinputlisting{code/03-tracer/example_code_post.txt}
Dieser ergibt den folgenden Trace:
\begin{align*}
  [&[signal(1, 2),\ signal(2, 3),\ signal(3, 4),\ pre(4, a?, default),\ post(5, default)]\\
  &[wait(8, 2),\ lock(9, 1, -, 1),\ pre(10, x!),\ post(16, 2, x!, 1),\ unlock(17, 1)]\\
  &[wait(11, 3),\ pre(12, y!),\ post(13, 3, y!, 1),\ pre(14, x?),\ post(15, 2, x?, 10, 1)]\\
  &[wait(6, 4),\ pre(7, y?),\ post(18, 3, y?, 12, 1)]]
\end{align*}
Aus diesem lässt sich der Ablauf des Programms vollständig rekonstruieren.
\extend{Tracer}


\section{Instrumenter}\label{Chap:Tracer-Sec:Instrumenter}
\draft{Instrumenter}
Um den Trace zu erzeugen, müssen verschiedene Operationen durch Funktionen
des Tracers ersetzt bzw. erweitert werden. Wie man an dem Beispiel 
unschwer erkennen kann, besitzt die Version mit dem Tracer einen deutlich längeren
Programmcode, was bei der Implementierung zu einer größeren Arbeitslast 
führen kann. Da sich der Tracer auch negativ auf die Laufzeit des Programms 
auswirken kann, ist es in vielen Situationen nicht erwünscht, ihn in den 
eigentlichen Release-Code einzubauen, sondern eher in eine eigenständige 
Implementierung, welche nur für den Tracer verwendet werden. Um dies zu
automatisieren wurde ein zusätzliches Programm implementiert, welches in der 
Lage ist, den Tracer in normalen Go-Code einzufügen. Die Implementierung, 
welche ebenfalls in~\cite{GoChan} zur Verfügung steht, arbeitet mit einem 
Abstract Syntax Tree. Bei dem Durchlaufen dieses Baums werden die 
entsprechenden Operationen in dem Programm erkannt, und durch ihre entsprechenden 
Tracer-Funktionen ersetzt bzw.\ ergänzt. Neben dem Ersetzen der verschiedenen 
Operationen werden außerdem einige Funktionen hinzugefügt. Zu Begin der 
Main-Funktion des Programms wird der Tracer initialisiert. Zusätzlich
wird eine zusätzliche Go-Routine gestartet, in welcher ein Timer läuft. 
Ist dieser abgelaufen, wird  die Analyse gestartet, 
auch wenn das Programm noch nicht vollständig durchlaufen ist. Dies führt dazu,
dass auch Programme, in welchen ein Deadlock aufgetreten ist, analysiert 
werden können. Endet das Programm in der vorgegebenen Zeit, wird der Analyzer 
nach der Beendigung des Programms gestartet.
\extend{Instrumenter}

\section{Laufzeit}\label{Chap:Tracer-Sec:Laufzeit}
\paragraph{Instrumenter} Zuerst soll die Laufzeit des Instrumenters betrachtet 
werden. Es ist erwartbar, 
dass sich die Laufzeit linear in der Anzahl der Ersetzungen in dem AST, also 
der Anzahl der Mutex- und Channel-Operationen verhält. Dies bestätigt sich auch durch 
die Messung der Laufzeit des Programms (vgl. 
Abb.~\ref{Chap:Tracer-Sec:Laufzeit-Img:LaufzeitInstrumenter})\\
\begin{minipage}{0.45\textwidth}
  \centering  
  \includegraphics[width=\textwidth]{img/tracer/Runtime_Instrumenter.eps}
  \captionof{figure}{Laufzeit des Instrumenters}
  \label{Chap:Tracer-Sec:Laufzeit-Img:LaufzeitInstrumenter}
\end{minipage}
\hfill
\begin{minipage}{0.45\textwidth}
  \centering
  \includegraphics[width=\textwidth]{img/tracer/Runtime_Tracer.eps}
  \captionof{figure}{Prozentualer Overhead des Tracers ohne Analyse}
  \label{Chap:Trace-Sec:Laufzeit-Img:LaufzeitTracer}
\end{minipage}
Der abgebildete Graph zeigt die Laufzeit des Programms in $s$ abhängig von der 
Größe des Programms. Das Programm besteht dabei aus einem Testprogramm, welches 
alle möglichen Situationen mit Channels und Mutexen abbildet. Die Vergrößerung 
des Programmes wurde dadurch erreicht, dass die Datei mit dem Programmcode 
mehrfach in dem Projekt vorkam. Ein Projekt mit Größe $n$ besteht vor der 
Instrumentierung also 
aus $n$ Dateien, mit insgesamt $65n$ Zeilen von Code und $52n$ Ersetzungen
in dem AST. Die tatsächliche Laufzeit des Instrumenters auf einen 
Programm hängt schlussendlich natürlich von der tatsächlichen Größe des 
Projekt und der Verteilung der Mutex- und Channel-Operationen in dem Code ab.\\
\begin{table}[!h]
  \centering
  \begin{tabular}{|c|c|c|c|c|}
  \hline
  Projekt & LOC & Nr. Dateien & Nr. Ersetzungen & Zeit {[}s{]} \\ \hline
  ht-cat & $733$ & $7$ & $233$ & $0.013 \pm 0.006$ \\ \hline
  go-dsp & $2229$ & $18$ & $600$ & $0.029 \pm 0.009$ \\ \hline
  goker & $9783$ & $103$ & $4928$ & $0.09 \pm 0.03$ \\ \hline
  \end{tabular}
  \caption{Laufzeit des Instrumenters für ausgewählte Programme}
  \label{Chap:Tracer-Sec:Laufzeit-Tab:LaufzeitInstrumenter}
\end{table}
Zusätzlich wurde die Messung auch mit drei tatsächlichen Programmen 
durchgeführt. Die dort gemessenen Werte befinden sich in 
Tabelle~\ref{Chap:Tracer-Sec:Laufzeit-Tab:LaufzeitInstrumenter}. Gerade in 
Abhängigkeit von der Anzahl der Ersetzungen, stimmen die hier gemessenen Werte
mit denen in Abb.~\ref{Chap:Tracer-Sec:Laufzeit-Img:LaufzeitInstrumenter} gut 
überein, während es bei den anderen Parametern größere Abweichungen gibt.
Dies bestätigt dass der dominante Faktor für die Laufzeit des Programms 
die Anzahl der Ersetzungen in dem AST ist, und die Laufzeit linear von dieser 
abhängt.
\paragraph{Tracer} Folgend soll nun auch die Laufzeit des Tracers betrachtet werden.
Hierbei wird nur die Laufzeit des eigentlichen Tracers, nicht aber der anschließenden 
Analyse betrachtet. Um den Overhead in Abhängigkeit von der Größe des Projektes messen 
zu können, wird das selbe Testprogramm betrachtet, welches bereits in der Messung 
für den Instrumenter verwendet wurde. Abb. \ref{Chap:Trace-Sec:Laufzeit-Img:LaufzeitTracer}
zeigt den gemessenen Overhead. Der durchschnittliche Overhead über alle gemessenen Werte 
liegt dabei bei $14 \pm 2\ \%$. Da der Overhead aber linear davon abhängt, 
wie größ der Anteil der Mutex- und Channel-Operationen im Verhältniss zu 
der Größe bzw. der Laufzeit des gesammten Programms ist, kann dieser Wert abhängig 
von dem tatsächlichen Programm start schwanken. Dies wird unteranderem klar, wenn man 
den Overhead für ht-cat ($9 \pm 3\ \%$) und go-dsp ($60\pm 18\ \%$) welche $51 \pm 21$ 
Prozentpunkte außeinander liegen. 
\extend{Laufzeit}


