\chapter{Analyze}\label{Chap:Analyze}
Das folgende Kapitel soll sich nun mit der Analyze des in Kap~\ref{Chap:Tracer} erstellten 
Trace befassen. Sie befasst sich dabei mit der erkennung potentieller Deadlocks 
durch die vorkommenden Mutexe und Routinen. \extend{Mehr zur Einführung}

\section{Deadlock durch Mutext}\label{Chan:Analyze-Sec:Mutex}
\draft{Deadlock Mutex Draft}
Als erstes sollen Deadlocks betrachtet werden, welche nur von (RW-)Mutexen erzeugt werden.
Dabei wird vor allem zyklischen Locking betrachtet, bei dem sich mehrere Routinen 
gegenseitig blockieren. 
Ein solches zyklisches Locking kann z.B. in der folgenden Situation auftreten. 
\lstinputlisting{code/04-analyzer/zyclic-locking-example.txt}
Routine 0 und Routine 1 können dabei gleichzeitig ausgeführt werden. Man betrachte den Fall, in dem 
Zeile 7 und 14 gleichzeitig ausgeführt werden, also Lock \texttt{y} von Routine 0 und Lock \texttt{x} 
von Routine 1 gehalten wird. In diesem Fall kann in keiner der Routinen die nächste Zeile augeführt werden,
da das jeweilige Locks, welches beansprucht werden soll bereits durch die andere Routine gehalten wird. 
Da sich diese Situation auch nicht von alleine auflößen kann, blockiert dass Programm, befindet sich also 
in einem zyklischen Deadlock.\\\\
Da solche Situtationen nur in ganz besonderen Situation auftreten (in dem obigen Beispiel müssen Zeilen 
7 und 14 genau gleichzeitig ausgeführt werden, ohne dass Zeile 8 oder 15 ausgeführt werden), muss 
ein Detektor, welcher vor solchen Situationen warnen soll, nicht nur tatsächliche Deadlocks, sondern
vor allem potenzielle, also nicht tatsächlich aufgetretene Deadlocks erkenne. Die Erkennung der 
potenziellen Deadlocks basiert hierbei auf iGoodLock~\cite{iGoodLock} und UNDEAD~\cite{Undead}. Dabei wird ein Lockgraph 
aufgebaut. Dieser Speichert die in dem Programm vorkommenden Knoten, sowie ihre Abhängigkeiten. 
Dies bedeutet, dass die Knoten des Graphen gerade die (RW-)Locks representieren. Es gibt dabei 
genau dann eine Kante von Knoten \texttt{x} nach \texttt{y}, wenn das Lock \texttt{y} beansprucht 
wird, während das Lock \texttt{x} gerade von der selben Routine gehalten wird. Eine 
genauere Erklärung der Implemenierung des Locks findet sich in~\cite{bachelor-project}. \todo{ist des erlabut, oder soll ich es nochmal komplett beschreiben}  \\\\
Der Graph wird basierend auf dem zufgezeichneten Trace aufgebaut. Dazu werden die Traces der 
einzelnen Routinen nacheinander durchlaufen. Für jede Routine erzeugen wir eine Liste \texttt{currentLocks} aller 
Locks, die momentan von der Routine gehalten werden. Die einzelnen Elemente des Trace einer 
Routine werden nun durchlaufen. Handelt es sich dabei um ein Lock Event eines Locks \texttt{x}, wird 
für jedes Lock \texttt{l} in \texttt{currentLocks} eine Kante von \texttt{l} nach \texttt{x} in den 
Lock-Graphen eingefügt. Anschließend wird \texttt{x} in \texttt{currentLocks} eingefügt.
Ist das handelt es sich bei dem Element um ein unlock Event auf dem Lock \texttt{x}, dann wird das 
letzt vorkommen von \texttt{x} auf \texttt{currentLocks} entfernt.\\
Nachdem der Trace einer Routine durchlaufen wurde, wird überprüft ob sich noch Elemente in 
\texttt{currentLocks} befinden. Ist dies der Fall, handelt es sich um Locks, welche zum Zeitpunkt der Terminierung 
des Programms noch nicht wieder freigegeben worden sind. Dies deutet darauf hin, dass die
entsprechende Routine nicht beendet wurde, z.B. weil das Programm bzw. die Main-Routine beendet wurden.
Dies kann einfach durch die entsprechende Logik des Programms zustande gekommen sein, es kann aber auch 
auf einen tatsächlich auftretenden Deadlock, z.B. durch doppeltes Locking des selben Locks in einer Routine, ohne dass 
es zwischenzeitlich wieder freigegeben wurde. In diesem Fall wird eine warnung ausgegeben.\\
Ein potenzieller Deadlock gibt sich nun, wenn in diesem Graph ein Kreis existert. Dabei muss darauf 
geachtet werden, dass nicht alle Kanten durch die selbe Routine erzeugt wurden, und dass in 
zwei, in dem Kreis hintereinander folgende Kanten der gemeinsame Knoten nicht beides mal durch eine 
R-Lock Operation durch Kanten verbunden wurde. Die Erkennung solcher Zyklen geschieht nun durch 
eine Tiefensuche auf dem erzeugen Baum. Wird ein solcher Zyklus erkannt, wird ebenfalls eine 
Warnung ausgegeben.
