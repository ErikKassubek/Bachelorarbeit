
\chapter{Implementierung} \label{Chap:Implement}

Das folgende Kapitel soll sich nun mit der Analyze des in Kap~\ref{Chap:Tracer} erstellten 
Trace befassen. Sie befasst sich dabei mit der Erkennung unerwünschter Situationen 
durch die vorkommenden Mutexe und Routinen. \extend{Mehr zur Einführung}







\section{Mutex} \label{Chap:Implement-Sec:Mutex}
\todo{Für mutex: regeln, gate-locks}







Eine 
genauere Erklärung der Implemenierung des Locks findet sich in~\cite{bachelor-project}. 
\todo{ist des erlabut, oder soll ich es nochmal komplett beschreiben}  \\\\
Die Lock-Bäume werden basierend auf dem aufgezeichneten Trace aufgebaut. Dazu werden die Traces der 
einzelnen Routinen nacheinander durchlaufen. Für jede Routine erzeugen wir eine Liste \texttt{currentLocks} aller 
Locks, die momentan von der Routine gehalten werden. Die einzelnen Elemente des Trace einer 
Routine werden nun durchlaufen. Handelt es sich dabei um ein Lock Event eines Locks \texttt{x}, wird 
für jedes Lock \texttt{l} in \texttt{currentLocks} eine Kante von \texttt{l} nach \texttt{x} in den 
Lock-Graphen eingefügt. Anschließend wird \texttt{x} in \texttt{currentLocks} eingefügt.
Ist das handelt es sich bei dem Element um ein unlock Event auf dem Lock \texttt{x}, dann wird das 
letzt vorkommen von \texttt{x} auf \texttt{currentLocks} entfernt.\\
Nachdem der Trace einer Routine durchlaufen wurde, wird überprüft ob sich noch Elemente in 
\texttt{currentLocks} befinden. Ist dies der Fall, handelt es sich um Locks, welche zum Zeitpunkt der Terminierung 
des Programms noch nicht wieder freigegeben worden sind. Dies deutet darauf hin, dass die
entsprechende Routine nicht beendet wurde, z.B. weil das Programm bzw. die Main-Routine beendet wurden.
Dies kann einfach durch die entsprechende Logik des Programms zustande gekommen sein, es kann aber auch 
auf einen tatsächlich auftretenden Deadlock, z.B. durch doppeltes Locking des selben Locks in einer Routine, ohne dass 
es zwischenzeitlich wieder freigegeben wurde. In diesem Fall wird eine warnung ausgegeben.\\
Ein potenzieller Deadlock gibt sich nun, wenn in diesem Graph ein Kreis existert. Dabei muss darauf 
geachtet werden, dass nicht alle Kanten durch die selbe Routine erzeugt wurden, und dass in 
zwei, in dem Kreis hintereinander folgende Kanten der gemeinsame Knoten nicht beides mal durch eine 
R-Lock Operation durch Kanten verbunden wurde. Die Erkennung solcher Zyklen geschieht nun durch 
eine Tiefensuche auf dem erzeugen Baum. Wird ein solcher Zyklus erkannt, wird ebenfalls eine 
Warnung ausgegeben.




\section{Channels}\label{Chap:Analyse-Sec:Channel}

\todo{cont, implementierung beschreiben}