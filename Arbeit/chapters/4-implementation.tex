
\chapter{Analyzer} \label{Chap:Implement}

Das folgende Kapitel soll sich nun mit der Implementierung der Analyze des in 
Kap.~\ref{Chap:Instrumenter-Sec:Trace} erstellten 
Trace befassen.

\section{Ablauf}
Um das Programm zu Analysieren wird der instrumentierte Code ausgeführt.
Besitzt es Select-Statements, dann wird es mehrfach durchgeführt, 
wobei jeweils eine andere, zufällige Ordnung für die Select-Cases 
betrachtet wird. Die betrachteten Ordnungen werden vor dem Durchlaufen 
der Programme erzeugt. Dabei kann vom Nutzer sowohl eine maximale 
Gesamtanzahl an Durchläufen angegeben werden. Zum anderen wird die 
Anzahl auch anhand der Dichte des Select-Cases beschränkt. Dazu wird 
eine neue, gültige Ordnung zufällig erzeugt und überprüft, ob diese 
Ordnung bereits zuvor erzeugt worden ist. Wenn sie neu ist, wird sie in 
die Liste der zu durchlaufenden Ordnungen eingefügt. Dabei wird gezählt,
wie oft eine Ordnung bereits zuvor erzeugt worden war. Auch für diesen 
Wert kann ein Maximalwert gesetzt werden.\\
Anschließend wird das Programm für jede der bestimmten Ordnungen analysiert.
Da das Programm gegebenenfalls mehrfach analysiert wird ist es wahrscheinlich, dass 
die selben Probleme mehrfach erkannt werden. Es ist daher nicht sinnvoll 
die Informationen direkt nach jedem Durchlauf auszugeben. Daher werden die 
Fehler gesammelt, und dabei jeweils gespeichert, bei welchen Durchlaufen die 
Fehler aufgetreten sind. Nach Abschluss aller Durchläufe werden die 
gesammelten Fehler vollständig ausgegeben.

\section{Aufzeichung des Trace}
Der Trace wird wie in Kap.~\ref{Chap:Instrumenter-Sec:Trace}
beschriebenen erzeugt. Er wird dabei durch ein Slice von Slices 
gespeichert, wobei jeder Slice den Trace einer Routine 
speichert. Dazu wird für jedes Trace-Element ein Struct definiert, 
welches alle notwendigen Informationen speichert. Über ein Interface 
wird dafür gesorgt, dass diese Elemente alle in das Slice des Trace eingefügt
werden kann. Dabei wird über einen Mutex dafür gesorgt, dass nicht 
zwei Routinen gleichzeitig ein Element in den Trace einfügen können\\
Jeder Routine wird ein fortlaufender Wert eines atomaren Zählers zugeordnet. 
Um die interne Id einer Routine zu erhalten, und auf die dem Trace zugeordnete 
Id zuzuordnen wird eine externe Bibliothek GoId~\cite{goid} verwendet.\\
Auch die Ids für Channels und Mutexes werden durch laufende, atomare Zähler 
implementiert, welche in den Objekten für Channels und Mutexes gespeichert werden.

\section{Mutex} \label{Chap:Implement-Sec:Mutex}
Um potenzielle Deadlocks durch Mutexe erkennen zu können werden nun, wie in 
Kap.~\ref{Chap:Back-Sec:Prob-SubSec:Mutex} beschrieben, Lock-Bäume verwendet.
Diese werden basierend auf dem aufgezeichneten Trace aufgebaut. Dazu werden die Traces der 
einzelnen Routinen nacheinander durchlaufen. Für jede Routine wird eine Liste \texttt{currentLocks} aller 
Locks erzeugt, die momentan von der Routine gehalten werden. Die einzelnen Elemente des Trace einer 
Routine werden nun durchlaufen. Handelt es sich dabei um ein Lock Event eines Locks \texttt{x}, 
wird eine s.g. \texttt{dependendency} erzeugt und gespeichert. Diese beinhaltet 
das Lock \texttt{x} sowie eine Liste aller von der Routine momentan gehaltenen 
Locks, dem s.g. \texttt{holdingSet hs}. Dieses entspricht gerade \texttt{currentLocks}. 
Diese \texttt{dependendency} stellt also 
eine Menge von Kanten von in dem Lockgraphen der Routine da. 
Anschließend wird \texttt{x} in \texttt{currentLocks} eingefügt.
Ist das handelt es sich bei dem Element um ein unlock Event auf dem Lock \texttt{x}, dann wird das 
letzte Vorkommen von \texttt{x} auf \texttt{currentLocks} entfernt.\\
Nachdem der Trace einer Routine durchlaufen wurde, wird überprüft ob sich noch Elemente in 
\texttt{currentLocks} befinden. Ist dies der Fall, handelt es sich um Locks, welche zum Zeitpunkt der Terminierung 
des Programms noch nicht wieder freigegeben worden sind. Dies deutet darauf hin, dass die
entsprechende Routine nicht beendet wurde, z.B. weil das Programm bzw. die Main-Routine beendet wurden.
Dies kann einfach durch die entsprechende Logik des Programms zustande gekommen sein, es kann aber auch 
auf einen tatsächlich auftretenden Deadlock, z.B. durch doppeltes Locking des selben Locks in einer Routine, ohne dass 
es zwischenzeitlich wieder freigegeben wurde. In diesem Fall wird eine Warnung ausgegeben.\\
Ein potenzieller Deadlock gibt sich nun, wenn in diesem aus den Bäumen zusammengesetzten 
Graph ein Zyklus existiert.
Nicht alle Zyklen bilden dabei gültige Zyklen. zum Beispiel muss darauf 
geachtet werden, dass nicht alle Kanten durch die selbe Routine erzeugt wurden, und dass in 
zwei, in dem Kreis hintereinander folgende Kanten der gemeinsame Knoten nicht beides mal durch eine 
R-Lock Operation durch Kanten verbunden wurde. Gültige Zyklen lassen sich durch 
dir folgenden Formeln charakterisieren.
\begin{align}
  &\forall\ i, j \in \{1,...,n\}\ \lnot (hs_i \cap hs_j = \emptyset) \rightarrow (i = j) 
  \tag{\ref{Chap:Implement-Sec:Mutex}.a}
  \label{Chap:Implement-Sec:Mutex.a}\\
  &\forall\ i \in \{1,...,n-1\}\ mu_i \in hs_{i+1} 
  \tag{\ref{Chap:Implement-Sec:Mutex}.b}
  \label{Chap:Implement-Sec:Mutex.b}\\
  &mu_n \in hs_{1} 
  \tag{\ref{Chap:Implement-Sec:Mutex}.c}
  \label{Chap:Implement-Sec:Mutex.c}\\
  &\forall i \in \{1,\ldots,n-1\}\ read(mu_i) \to (\forall mu\in hs_{i+1} (mu = mu_i) \to \lnot read(mu))
  \tag{\ref{Chap:Implement-Sec:Mutex}.d}
  \label{Chap:Implement-Sec:Mutex.d}\\
  &read(mu_n) \rightarrow 
  (\forall\ mu \in hs_{1}: (mu = mu_n) \to \lnot read(mu))
  \tag{\ref{Chap:Implement-Sec:Mutex}.e}
  \label{Chap:Implement-Sec:Mutex.e}\\
  &\makecell{\forall\ i, j \in \{1,...,n\}\ \lnot (i = j) \rightarrow 
  (\exists\ mu_1 \in hs_i\ \exists\ mu_2 \in hs_j ((mu_1 = mu_2) \rightarrow\\
  (read(mu_1) \land read(mu_2))))\phantom{123123123122131231231231223123}}
  \tag{\ref{Chap:Implement-Sec:Mutex}.f}
  \label{Chap:Implement-Sec:Mutex.f}
\end{align}
Dabei bezeichnet $mi_i$ den Mutex $hs_i$ das holdingSet der $i$-ten 
in dem Zyklus.\ \eqref{Chap:Implement-Sec:Mutex.a} stellt sicher, dass das selbe 
Lock nicht in dem HoldingSet von zwei verschiedenen Routinen auftauchen kann.\
\eqref{Chap:Implement-Sec:Mutex.b} und~\eqref{Chap:Implement-Sec:Mutex.c}
sorgen dafür, dass es sich bei der Kette tatsächlich um einen Zyklus handelt, 
dass also das Lock einer Dependency immer in dem HoldingSet
der nächsten Dependency enthalten ist und das Lock der letzten Routine 
wiederum im HoldingSet der ersten Routine liegt, um den Zyklus zu schließen.
\eqref{Chap:Implement-Sec:Mutex.d} bis \eqref{Chap:Implement-Sec:Mutex.f} 
beschäftigen sich mit dem Einfluss von RW-Locks auf die Gültigkeit von 
Zyklen. Auch wenn \eqref{Chap:Implement-Sec:Mutex.a} bid~\eqref{Chap:Implement-Sec:Mutex.c} 
erfüllt sind, ist dies dennoch keine gültige Kette, wenn sowohl
der Mutex $mu_i$ als auch der Mutex $mu$ in $hs_{i+1}$, 
für die $mu = mu_i$ gilt, beides Reader-Locks
sind, also Locks welche durch eine RLock-Operation erzeugt worden sind. 
Dass solche Pfade ausgeschlossen werden wird durch \eqref{Chap:Implement-Sec:Mutex.d} 
und \eqref{Chap:Implement-Sec:Mutex.e} sichergestellt.
\eqref{Chap:Implement-Sec:Mutex.f} beschäftigt sich mit Gate-Locks. 
Dabei handelt es sich um Situationen, bei denen mehrere Teile des Programmcodes,
welche zu einem Deadlock führen könnten durch ein Lock umschlossen sind, 
in der Praxis also nicht gleichzeitig ausgeführt werden und somit einen Deadlock 
verhindern.
Die Regel besagt nun, dass wenn es einen Mutex gibt,
der in den HoldingSets zweier verschiedener Dependencies in dem Pfad vorkommt, so müssen
beide diese Mutexe Reader-Locks sein. Sind sie es nicht, handelt es sich um Gate-Locks, und
der entsprechende Pfad kann somit nicht zu einem Deadlock führen.\\\\
Für die Suche nach solchen Zyklen wird eine Depth-First-Search auf den gesammelten 
Dependencies ausgeführt. Dazu wird zuerst eine Dependency auf einen Stack gelegt. 
Der Stack entspricht immer dem momentan
betrachteten Pfad. Anschließend werden schrittweise weitere Dependencies auf den 
Stack gelegt, wobei darauf geachtet wird, dass aus jeder Routine immer nur maximal 
eine Dependency auf dem Stack liegt. Bevor eine Dependency zu dem Stack 
hinzugefügt wird, wird überprüft ob der durch den Stack betrachtete Pfad 
einen gültigen Pfad bilden würde, ob also \eqref{Chap:Implement-Sec:Mutex.a}, 
\eqref{Chap:Implement-Sec:Mutex.b}, \eqref{Chap:Implement-Sec:Mutex.d} und 
\eqref{Chap:Implement-Sec:Mutex.f} immer noch gelten würden. Ist dies nicht
der Fall, so wird die Dependency nicht auf den Stack gelegt. Werden die 
Regeln hingegen erfüllt, dann wird überprüft, um der Stack nun einen gültigen
Zyklus enthält, also auch \eqref{Chap:Implement-Sec:Mutex.c} und \eqref{Chap:Implement-Sec:Mutex.e}
gültig sind. In diesem Fall wurde ein potenzielles Deadlock gefunden, und dies 
ausgegeben. Andernfalls werden weiter Dependency auf dem Stack hinzugefügt. 
Dies wird wiederholt, bis es keine Dependency mehr gibt, die auf den Stack VaT
gelegt werden könnte. In diesem Fall werden per Backtracking Dependencies 
von dem Stack entfernt, so dass andere Pfade ausprobiert werden können.
Dies wird so lange durchgeführt. Bis alle gültigen Kombinationen durchprobiert 
worden sind.

\section{Channels}\label{Chap:Analyse-Sec:Channel}
Die Analyse des Programs zur Erkennung und Beschreibung von durch Channels ausgelösten 
Problemen läuft in mehreren Schritten ab. Zuerst werden die Vectorclock-Informationen 
der einzelnen Operationen bestimmt und mit diesen ein vectorclock-annotated Trace 
(VaT) erzeugt. Basierend auf diesen sucht der Analyzer nach potenziellen 
Situationen, welche zu blockenden Message-Bugs oder nicht gelesene
Nachrichten auf gebufferten Channels führen können. Zum Schluss 
wird nach Situationen gesucht, bei denen es zu einem Send auf einen 
geschlossenen Channel kommen kann, da solche Situationen zu Laufzeitfehlern 
führen, welche den Abbruch eines Programms zur Folge haben.
Receives auf geschlossenen Channels werden nicht betrachtet, da diese 
lediglich einen Null-Wert zurückgeben und nicht blocken, somit also nicht 
zu Laufzeitfehlern führen. 

\paragraph{Bestimmung des vectorclock-annotierten Trace VAT}
Basierend auf dem aufgezeichneten Trace soll nun ein vectorclock-annotierter Trace 
(VAT) erzeugt werden. Dieser besteht aus einer Reihe von \texttt{vcn}'s, 
welche jeweils eine Send-, Receive- oder Close-Operation repräsentieren.
Andere Operation, wie z.B. signal-wait werden zwar bei der Berechnung der Vectorclocks 
beachtet, allerdings nicht in den VAT aufgenommen, da sie für die weitere 
Analyse nicht benötigt werden. Ein \texttt{vcn} beinhaltet dabei die Channel-Id,
die Routine, ob es sich um Send- oder Receive handelt (bei Close beliebig gesetzt),
ein Counter für die Anzahl der bereits erfolgreich abgeschlossenen Send- bzw.
Receive-Operationen bei der Ausführung des Send- oder Receive (bei Close -1), 
die Position der Operation im Programmcode
sowie die Pre- und Post-Vectorclocks der Operation. Eine Close Operation 
wird dabei dadurch erkannt, dass die Pre- und Post-Vectorclocks übereinstimmen.\\\\
Bevor der eigentliche VAT erzeugt wird werden erst die Vectorclocks zu allen
Zeitpunkten bestimmt. Dazu wird für jede Routine eine Vectorclock initialisiert. 
Anschließend werden die Elemente in der Reihenfolge durchlaufen, in der sie 
in den Trace eingefügt wurden, also aufsteigend sortiert nach dem Timestamp 
der Trace-Elemente. Für jeden Zeitpunkt wird nun eine Vectorclock berechnet,
wobei für jeden Timestamp immer diejenige Vectorclock gespeichert wird, die 
der Vectorclock entspricht, auf welcher die entsprechende Operation ausgeführt wurde.
Für die Berechnung der Vectorclocks werden signal-wait Paare wie das Senden 
einer Nachricht von signal nach wait betrachtet.\\
Für send (post) und Signal bzw. Receive (post) und Wait werden nun die Vectorclocks 
aktualisiert.
Für Send und Signal wird lediglich der eigene Timestamp in der eigenen 
Vectorclock um eins erhöht. Für die Aktualisierung bei einem Receive oder Wait 
wird die Vectorclock zur Zeit von Send oder Signal benötigt. Da diese in jedem 
Fall vor dem Receive oder Wait erzeugt worden sind, wurden sie bereits berechnet.
Da für die Receive-Elemente in dem Trace die Zeitstempel der Send-Operationen 
gespeichert sind, ist eine eindeutige Zuordnung der Send- und Receive-Statements 
möglich. Für die Signal- und Wait-Elemente ist jeweils die Id der neu erzeugten 
Routine gespeichert. Es ist also auch hier eine eindeutige Zuordnung möglich. 
Die Vectorclock der Send- bzw. Signal-Operation kann also immer eindeutig bestimmt 
werden und die Vectorclock somit wie in Kap.~\ref{Chap:Analyze-Sec:Channel-SubSec:Dangling}
beschrieben aktualisiert werden.\\
Für alle anderen Element, also alle Pre-Elemente, Close-Operationen und Mutex-Operationen 
werden die Vectorclocks lediglich kopiert.\\\\
Nach der Berechnung der Vectorclocks kann nun der VAT bestimmt werden. 
Dazu wird nun der Trace für die einzelnen Routinen durchlaufen. Bei jedem 
Pre- und PreSelect-Element wird eine \texttt{vcn} erzeugt. Dazu wird der restliche Trace 
der selben Routine durchlaufen um das zugehörige Post-Element zu finden. 
Wird diese gefunden wird 
das \texttt{vcn} erzeugt, wobei die Pre- und Post-Vectorclock über den 
Zeitstempel der Pre- und Post-Elemente aus der Liste der Vectorclocks 
übernommen wird.
Wird kein Post-Element gefunden, handelt es sich also um ein hängendes Event, 
wird die Pre-Vectorclock auf die Vectorclock des Pre-Events und alle Elemente 
der Post-Vectorclock auf \texttt{maxInt}, als den maximal möglichen Wert 
gesetzt. In diesem Fall wird der entsprechende Channel außerdem in die Liste der
hängenden Channels aufgenommen. 
\\
Für Close-Elemente gibt es nur ein Element in dem Trace. Aus diesem Grund 
besitzt das Element nur eine Vectorclock. Pre- und Post-Vectorclock werden 
dabei auf die gleiche Vectorclock gesetzt.

\paragraph{Erkennung pottenzieller Communication-Bugs}
Basierend auf dem VAT können nun tatsächlich aufgetretene oder potenzielle 
Communication-Bugs erkannt werden. Dabei wird nach Channel-Operationen 
gesucht, welche bei bestimmten Abläufen keine gültigen Kommunikationspartner
besitzen. Dies führt zu einem blocken Bug, durch das warten auf 
Sends oder Receives oder das Senden von Nachrichten auf gebufferten 
Channels, ohne das die Nachricht jemals ausgelesen wird.\\  
Zuerst werden basierend auf dem VAT alle potenziellen Kommunikationspartner 
für Send-Receive Paare bestimmt.
Zur Suche nach alternativen Kommunikationspartner von ungebufferten Channels
werden alle Kombinationen
von zwei Elementen in dem VAT betrachtet. Dabei werden all diejenigen 
Elemente verglichen, bei welchen beide Operationen auf dem selben Channel 
ausgeführt werden und eines der 
Elemente ein Send- und das andere eine Receive-Operation ist. Zwei Operationen
werden als potenzielle Kommunikationspartner angesehen, wenn die Pre- oder
Vectorclocks der beiden 
Operationen unvergleichbar sind.\\
Für den gebufferten Channel müssen Send- und Receive nicht 
gleichzeitig ausgeführt werden. Aus diesem Grund müssen die Vectorclocks 
nicht unvergleichbar sein. Um mögliche Kommunikationspartner zu erkennen 
wird daher vor der Betrachtung aller möglicher Kombination
die in Abschnitt \ref{Chap:Theo-Sec:Analyze-SubSec:Channel} beschriebenen 
Werten für jedes Element in dem VAT ermittelt und in dem VAT-Element abgespeichert.
Bei dem Vergleich wird bei gebufferten Channels nun überprüft, 
ob Die Formeln \eqref{Form:1} und \eqref{Form:2} aus \ref{Chap:Theo-Sec:Analyze-SubSec:Channel}
erfüllt sind. In diesem Fall wird angenommen, dass die beiden Operationen
miteinander Kommunizieren können.\\
Basierend auf diesen möglichen Kommunikationspartnern werden nun alle möglichen
Kommunikationsabläufe betrachtet. Für die Betrachtung aller Abläufe 
wird Backtracking verwendet. Zuerst wird schrittweise rekursiv für jede Send-Operation 
in dem Trace eine potenzielle Receive-Operation als Kommunikationspartner 
in einen partiellen Kommunikationsablauf eingefügt, sofern die 
Receive-Operation einen gültigen Kommunikationspartner bildet. Ein Receive 
ist ein potenzieller Kommunikationspartner, wenn eine Kommunikation 
basierend auf dem VAT möglich ist, die Receive-Operation noch von keiner 
anderen Send-Operation als Partner verwendet wird. Besitzt ein solcher Pfad 
einen Kommunikationspartner für jedes Send-Statement in dem Trace, bedeutet dies, 
dass der entsprechende Ablauf nicht zu einem Bug führen kann. Ist es hingegen 
nicht möglich einem Send ein gültiges Receive zuzuordnen, dann führt der 
entsprechende Ablauf zu einem potenziellen Bug. Bevor dieser zurückgegeben 
werden kann muss erst noch überprüft werden, ob es sich um eine 
gültige Ausführungsordnung handelt, wie in Abschnitt~\ref{Chap:Theo-Sec:Analyze-SubSec:Channel} 
beschrieben. Ist diese der Fall, wird eine entsprechende Nachricht ausgegeben, 
welche die Operation, welche zu dem Bug führt, sowie den partiellen
Ausführungspfad, bei welchem der Bug auftreten kann, enthält.\\In beiden 
Fällen, in denen das Hinzufügen weiterer Send-Operationen nicht mehr möglich ist 
wird nun Backtracking verwendet, um die weiteren Ausführungspfade zu betrachten.
Da die Reihenfolge, mit welcher die Send-Statements in den partiellen 
Ausführungspfad eingefügt werden werden, insbesondere welche Operation als erstes 
betrachtet wird, Auswirkung auf die Detektion von Problemen haben kann, 
wird das ganze für jede zyklische Permutation der Send-Operationen wiederholt. \\
Das Ganze wird anschließenden ein zweites Mal ausgeführt, wobei die Rollen 
von Send und Receive hierbei vertauscht sind. Es wird also für jedes Receive 
eine Send-Operation gesucht. 

\paragraph{Erkennung von potenziellem Senden auf geschlossenem Channel}
Für die Suche nach Situationen, die dazu führen können, dass auf einem 
geschlossenen Channel gesendet wird, werden die Vectorclock aller in dem 
Trace vorhandenen Close-Operationen mit den Vectorclocks aller Send-Operationen
auf dem selben Channel verglichen. Ist mindestens eine der beiden Vectorclocks 
unvergleichbar, dann nimmt das Programm an, das eine Send-Operation auf 
einen geschlossenen Channel möglich ist, und gibt eine entsprechende 
Warnung zurück. Es kommt auch zu einem Send auf einem geschlossenen Channel, 
wenn die Vectorclock der Close-Operation streng vor den Vectorclocks 
der Send-Operation sind. In diesem Fall kommt es aber in jedem Fall zu einem 
Send auf einen geschlossenen Channel und damit zu einem Laufzeitfehler, 
welcher durch den Detektor aufgefangen und erkannt wird.


\section{Beispiel}
In Anhang~\ref{Appendix-2} findet sich ein Beispielprogramm inklusive
originalem und instrumentierten Code, der neuen Main-Datei und
dem erhalten Ausgabe. 