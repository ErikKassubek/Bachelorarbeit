\chapter{Zusammenfassung}\label{chap:conclusion}
Ziel dieser Arbeit war es einen Detektor für von Mutexen 
und Channels erzeugte Concurrency-Bugs zu entwickeln und zu implementieren.\\
Der implementierte Detektor vereinigt und erweitert dabei verschiedene Methoden 
zur Erkennung solcher Probleme.\\
Der Tracer und Analyzer wird dabei in den originalen Programmcode eingefügt.
Dazu wurde ein Programm entwickelt, welches in der Lage ist diese Instrumentierung
automatisch auszuführen.\\
Der Detektor erzeugt dynamisch einen Trace 
eines vorliegenden Programms, welcher im Anschluss analysiert werden kann.\\
Für Deadlocks durch Mutexe werden dabei Lock-Bäume zur Erkennung von zyklischem 
Locking verwendet, sowie nach doppeltem Locking gesucht.\\
Für Communication-Bugs, welche durch Mutexe erzeugt werden
werden Pre- und Post-Vectorclocks bestimmt, welche eine Berechnung
von potenziellen Kommunikationspartnern ermöglicht. Mit diesen lässt sich 
durch die Unordnung der Kommunikationspartner nach Programmabläufen suchen, 
bei welchen Kommunikationsoperationen kein gültigen Kommunikationspartner 
besitzen und daher den Code blockieren, oder bei gebufferten Channels 
Nachrichten versenden, welche nie gelesen werden. \\
Mit Hilfe der Vectorclocks lassen sich außerdem Situationen erkennen, 
in welchen ein Send auf einen geschlossenen Channel passieren könnte.\\
Zur Betrachtung von verschiedenen durch Selects verzweigte Ablaufpfade
kann das Programm mehrfach durchlaufen werden, wobei jeweils ein 
anderer Ablauf bevorzugt werden kann.

Bei der Anwendung des Detektors auf konstruierte und tatsächliche Programme 
ist er in der Lage gut $92\%$ aller betrachteten 
Situationen richtig zu erkennen. 