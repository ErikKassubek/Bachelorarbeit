\chapter{Zusammenfassung}\label{chap:conclusion}
Ziel dieser Arbeit war es einen Detektor für von Mutexen 
und Channels erzeugte Concurrency-Bugs zu entwickeln und zu implementieren.\\
Der implementierte Detektor vereinigt und erweitert dabei verschiedene Methoden 
zur Erkennung solcher Probleme. Der Detektor erzeugt dynamisch einen Trace 
eines vorliegenden Programms, welcher im Anschluss analysiert werden kann.
Für Deadlocks durch Mutexe werden dabei unter anderem Lock-Bäume verwendet. 
Für Channel-Operationen wird der Trace mit Vector-Clocks erweitert, 
durch welche Schlussfolgerungen auf mögliche Kommunikationspartner oder 
das potenzielle Senden auf geschlossene Channels erkannt werden. 
Bei der Anwendung auf konstruierte Probleme und tatsächliche Programme 
ist der so entwickelte Detektor in der Lage etwa $88.5\%$ aller betrachteten 
Situationen richtig zu kategorisieren. 