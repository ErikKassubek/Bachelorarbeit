\appendix
\chapter{Beschreibung der betrachteten Programme}\label{Appendix-1}
\begin{table}[h]
  \centering
  \begin{tabular}{|l|c|}
  \hline
  \textbf{Beschreibung} & \textbf{Ok?} \\ \hline
  Potenzielles Deadlock durch zyklisches Locking von zwei Mutexen & Ja \\ \hline
  Potenzielles Deadlock durch zyklisches Locking von drei Locks & Ja \\ \hline
  Locking von zwei Locks, welche keinen Deadlock bilden, da Locking nicht zyklisch ist & Ja \\ \hline
  Zyklisches Locking welches durch Gate-Locks nicht zu einem Deadlock führen kann & Ja \\ \hline
  \makecell[l]{Potenzielles Deadlock, welches durch Verschachtlung mehrerer\\Routinen (fork/join) verschleiert wird} & Nein \\ \hline
  Deadlock durch doppeltes Locken & Ja \\ \hline
  Tatsächliches Deadlock durch zyklisches Locking von Locks in zwei Routinen & Ja \\ \hline
  Tatsächliches Deadlock durch zyklisches Locking von Locks in drei Routinen & Ja \\ \hline
  Doppeltes Locking mit TryLock (TryLock $\to$ Lock) & Ja \\ \hline
  Kein doppeltes Locking mit TryLock (Lock $\to$ TryLock) & Ja \\ \hline
  Deadlock durch zyklisches Locking mit TryLock & Ja \\ \hline
  Zyklisches Locking, welches durch TryLock nicht zu einem Deadlock führen kann & Ja \\ \hline   
  \makecell[l]{Potenzielles Deadlock mit RW-Mutexe in zwei Routinen\\(R1: x.RLock $\to$ y.Lock, R2: y.Lock $\to$ x.Lock)} & Ja \\ \hline
  \makecell[l]{Potenzielles Deadlock mit RW-Mutexe in zwei Routinen\\(R1: x.RLock $\to$ y.Lock, R2: y.RLock $\to$ x.Lock)} & Ja \\ \hline
  \makecell[l]{Kein potenzielles Deadlock mit RW-Mutexe in zwei Routinen\\(R1: x.RLock $\to$ y.RLock, R2: y.RLock $\to$ x.RLock)} & Ja \\ \hline
  \makecell[l]{Kein potenzielles Deadlock mit RW-Mutexe in zwei Routinen\\(R1: x.Lock $\to$ y.RLock, R2: y.RLock $\to$ x.Lock)} & Ja \\ \hline
  \makecell[l]{Kein potenzielles Deadlock mit RW-Mutexe in zwei Routinen\\(R1: x.RLock $\to$ y.Lock, R2: y.RLock $\to$ x.RLock)} & Ja \\ \hline
  \makecell[l]{Kein potenzielles Deadlock mit RW-Mutexe in zwei Routinen\\(R1: x.Lock $\to$ y.RLock, R2: y.RLock $\to$ x.RLock)} & Ja \\ \hline
  Kein potenzielles Deadlock, wegen Lock von RW-Locks als Gate-Locks & Ja \\ \hline
  Potenzielles Deadlock, da R-Lock von Deadlock nicht als Gate-Lock funktioniert & Ja\\ \hline
  \makecell[l]{Doppeltes Locking von RW-Locks, welches zu Deadlock führt\\(Lock$\to$Lock, RLock$\to$Lock, Lock$\to$Rlock)} & Ja \\ \hline
  \makecell[l]{Doppeltes Locking von RW-Locks, welches nicht zu einem\\Deadlock führt(RLock$\to$Rlock)} & Ja \\ \hline
  \end{tabular}
  \caption{Beschreibung der für die Auswertung betrachteten 
    Standardprogramme zur Erkennung von Situationen mit Mutexen, sowie 
    Information (Ok?) ob die Situation korrekt erkannt wurde. Ja bedeutet, dass 
    die Situation korrekt erkannt wurde
    und Nein, das das Problem nicht richtig kategorisiert wurde.}
  \label{App-Stand-Mut}
  \end{table}

\begin{table}[h]
  \centering
  \begin{tabular}{|l|c|}
  \hline
  \textbf{Beschreibung} & \textbf{Ok?} \\ \hline
  Deadlock oder hängende Routine durch Receive auf ungepuffertem Channel ohne Send & Ja \\ \hline
  Deadlock oder hängende Routine durch Receive auf gepuffertem Channel ohne Send & Ja \\ \hline
  Deadlock oder hängende Routine durch Send auf ungepuffertem Channel ohne Receive & Ja \\ \hline
  \makecell[l]{Kein Deadlock aber ungelesene Nachricht in Channel durch Send in \\Main-Routine auf gepuffertem Channel ohne Receive} & Ja \\ \hline
  \makecell[l]{Deadlock durch zweifaches Send auf gebufferten Channel in Kapazität 1\\in Main-Routine ohne Receive} & Ja \\ \hline
  \makecell[l]{Ungelesene Nachricht bei [$R_0$: \{$c_1^1$<-1; $c_1^1$<-1; $c_1^1$<-1\} , $R_1$: \{<-$c_1^1$\}, $R_2$: \{<-$c_1^1$\}]\\sowie Erkennung der
    potenziellen Kommunikationspartner} & Ja \\ \hline
  \makecell[l]{Keine Probleme bei [$R_0$: \{$c_1^1$<-1; $c_1^1$<-1\} , $R_1$: \{<-$c_1^1$\}, $R_2$: \{<-$c_1^1$\}]} & Ja \\ \hline
  Tatsächliches Send auf geschlossenen Channel & \todo{Ja} \\ \hline
  Potenzielles aber nicht tatsächliches Send auf geschlossenen Channel Channel & Ja \\ \hline
  \makecell[l]{Kein Problem, wenn Channel erst nach letztem Send geschlossen werden\\kann [$R_0$: \{$c_1^0$ <- 1; close($c_1^1$)\}, $R_1$: \{<-$c_1^1$\}]} & Ja \\ \hline
  \makecell[l]{Kein Problem, wenn Channel erst nach letztem Send geschlossen werden\\kann [$R_0$: \{<-$c_1^0$; close($c_1^1$)\}, $R_1$: \{$c_1^1$<-1\}]} & Nein \\ \hline
  \makecell[l]{Korrekte Kommunikationspartner bei [$R_0$: \{$c_1^1$ <- 1; $c_1^1$ <- 1; $c_1^1$ <- 1\}, $R_1$: \{<-x; <-x\}]\\(letztes Send hat keinen Kommunikationspartner)} & Ja \\ \hline
  \end{tabular}
  \caption{Beschreibung der für die Auswertung betrachteten 
  Standardprogramme zur Erkennung von Situationen mit Mutexen, sowie 
  Information (Ok?) ob die Situation korrekt erkannt wurde. Ja bedeutet, dass 
  die Situation korrekt erkannt wurde
  und Nein, das das Problem nicht richtig kategorisiert wurde. In den Fällen in denen 
  Teile des Programmcodes angegeben sind sei $R_0$ die Main-Routine. $c_i^j$
  sei ein Channel mit Kapazität $j$.}
\label{App-Stand-Mut}
\end{table}
\todo{tatsächliches send auf closed channel}