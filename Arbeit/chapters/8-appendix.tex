\appendix
\chapter{Beschreibung der betrachteten Programme}\label{Appendix-1}

\begin{longtable}[h]{|l|l|l|c|}
  \hline
  \textbf{Id} & \textbf{Type} & \textbf{SubType} & \textbf{Ok?} \\ \hline
  1 & Resource Deadlock & Zyklisches Locking & Ja \\ \hline
  2 & Resource Deadlock & Zyklisches Locking & Ja \\ \hline
  3 & Resource Deadlock & Zyklisches Locking & Ja \\ \hline
  4 & Resource Deadlock & Zyklisches Locking & Ja \\ \hline
  5 & Resource Deadlock & Zyklisches Locking & FN \\ \hline 
  7 & Resource Deadlock & Zyklisches Locking & Ja \\ \hline
  8 & Resource Deadlock & Zyklisches Locking & Ja \\ \hline
  11 & Resource Deadlock & Zyklisches Locking & Ja \\ \hline
  12 & Resource Deadlock & Zyklisches Locking & Ja \\ \hline
  13 & Resource Deadlock & Zyklisches Locking & Ja \\ \hline
  14 & Resource Deadlock & Zyklisches Locking & Ja \\ \hline
  15 & Resource Deadlock & Zyklisches Locking & Ja \\ \hline
  16 & Resource Deadlock & Zyklisches Locking & Ja \\ \hline
  6 & Resource Deadlock & Doppeltes Locking & Ja \\ \hline
  9 & Resource Deadlock & Doppeltes Locking & Ja \\ \hline
  10 & Resource Deadlock & Doppeltes Locking & Ja \\ \hline
  17 & Resource Deadlock & Doppeltes Locking & Ja \\ \hline
  18 & Resource Deadlock & Doppeltes Locking & Ja \\ \hline
  19 & Communication Deadlock & Channel & Ja \\ \hline
  20 & Communication Deadlock & Channel & Ja \\ \hline
  21 & Communication Deadlock & Channel & Ja \\ \hline
  22 & Communication Deadlock & Channel & Ja \\ \hline
  23 & Communication Deadlock & Channel & Ja \\ \hline
  24 & Communication Deadlock & Channel & Ja \\ \hline
  25 & Communication Deadlock & Channel & Ja \\ \hline
  26 & Communication Deadlock & Channel & Ja \\ \hline
  27 & Communication Deadlock & Channel & Ja \\ \hline
  28 & Communication Deadlock & Channel & Ja \\ \hline
  29 & Communication Deadlock & Channel & Ja \\ \hline
  30 & Communication Deadlock & Channel & Ja \\ \hline
  31 & Communication Deadlock & Channel & Ja \\ \hline
  32 & Communication Deadlock & Channel & Ja \\ \hline
  33 & Communication Deadlock & Channel & Ja \\ \hline
  34 & Communication Deadlock & Channel & Ja \\ \hline
  35 & Communication Deadlock & Channel & FP \\ \hline
  36 & Communication Deadlock & Channel & Ja \\ \hline
  37 & Mixed Deadlock & Channel\&Mutex & Ja \\ \hline
  38 & Mixed Deadlock & Channel\&Mutex & Ja \\ \hline
  39 & Mixed Deadlock & Channel\&Mutex & FP \\ \hline
  40 & Mixed Deadlock & Channel\&Mutex & Ja \\ \hline
  41 & Mixed Deadlock & Channel\&Mutex & FP \\ \hline
  42 & Mixed Deadlock & Channel\&Mutex & Ja \\ \hline
  43 & Mixed Deadlock & Channel\&Mutex & Ja \\ \hline
  \caption{Ergenisse für die betrachteten Standartsituationen. Ja bedeutet, 
  dass das Programm korrekt klassifiziert wurde. Bei FN handelte es sich 
  um ein False-Negative und bei FP um ein False-Positive. Die Beschreibungen
  für die Situationen befinden sich in Tab.~\ref{App-Stand-Des}}
  \label{App-Stand-Res}
\end{longtable}



\begin{longtable}[h]{|l|l|c|}
  \hline
  \textbf{Id} & \textbf{Beschreibung} & \textbf{Ok?} \\ \hline
  1 & Potenzielles Deadlock durch zyklisches Locking von zwei Mutexen & Ja \\ \hline
  2 & Potenzielles Deadlock durch zyklisches Locking von drei Locks & Ja \\ \hline
  3 & \makecell[l]{Locking von zwei Locks, welche keinen Deadlock bilden, da Locking nicht\\zyklisch ist} & Ja \\ \hline
  4 & \makecell[l]{Zyklisches Locking welches durch Gate-Locks nicht zu einem Deadlock\\führen kann} & Ja \\ \hline
  5 & \makecell[l]{Potenzielles zyklisches Deadlock, welches durch Verschachtlung mehrerer\\Routinen (fork/join) verschleiert wird} & FN \\ \hline
  6 & Tatsächliches Deadlock durch zyklisches Locking von Mutexen in zwei Routinen & Ja \\ \hline
  7 & Tatsächliches Deadlock durch zyklisches Locking von Mutexen in drei Routinen & Ja \\ \hline
  8 & Deadlock durch zyklisches Locking mit TryLock & Ja \\ \hline
  9 & Zyklisches Locking, welches durch TryLock nicht zu einem Deadlock führen kann & Ja \\ \hline   
  10 & \makecell[l]{Potenzielles Deadlock mit RW-Mutexe in zwei Routinen} & Ja \\ \hline
  11 & \makecell[l]{Kein potenzielles Deadlock mit RW-Mutexe in zwei Routinen} & Ja \\ \hline
  12 & Kein potenzielles Deadlock, wegen Lock von RW-Locks als Gate-Locks & Ja \\ \hline
  13 & Potenzielles Deadlock, da R-Lock von Deadlock nicht als Gate-Lock funktioniert & Ja\\ \hline
  14 & Deadlock durch doppeltes Locken & Ja \\ \hline
  15 & Doppeltes Locking mit TryLock (TryLock $\to$ Lock) & Ja \\ \hline
  16 & Kein doppeltes Locking mit TryLock (Lock $\to$ TryLock) & Ja \\ \hline
  17 & \makecell[l]{Doppeltes Locking von RW-Locks, welches zu Deadlock führt\\(Lock$\to$Lock, RLock$\to$Lock, Lock$\to$Rlock)} & Ja \\ \hline
  18 & \makecell[l]{Doppeltes Locking von RW-Locks, welches nicht zu einem\\Deadlock führt(RLock$\to$Rlock)} & Ja \\ \hline
  19 & \makecell[l]{Deadlock oder hängende Routine durch Receive auf ungepuffertem Channel\\ohne Send} & Ja \\ \hline
  20 & \makecell[l]{Deadlock oder hängende Routine durch Receive auf gepuffertem Channel\\ohne Send} & Ja \\ \hline
  21 & \makecell[l]{Deadlock oder hängende Routiune durch 2-faches Receive mit 1-fachem Send\\$[$$R_0$: \{$\leftarrow$$x_1^0$\}, $R_1$: \{$x_1^0$$\leftarrow$1\}, $R_2$: \{$\leftarrow$$x_1^0$\}$]$} & Ja \\ \hline
  22 & \makecell[l]{Deadlock oder hängende Routiune durch 2-faches Send mit 1-fachem Receive\\$[$$R_0$: \{$x_1^0$$\leftarrow$1\}, $R_1$: \{$x_1^0$$\leftarrow$1\}, $R_2$: \{$\leftarrow$$x_1^0$\}$]$} & Ja \\ \hline
  23 & \makecell[l]{Deadlock oder hängende Routine durch Send auf ungepuffertem Channel\\ohne Receive} & Ja \\ \hline
  24 & \makecell[l]{Kein Deadlock aber ungelesene Nachricht in Channel durch Send in \\Main-Routine auf gepuffertem Channel ohne Receive} & Ja \\ \hline
  25 & \makecell[l]{Deadlock durch zweifaches Send auf gebufferten Channel in Kapazität 1\\in Main-Routine ohne Receive} & Ja \\ \hline
  26 & \makecell[l]{Ungelesene Nachricht bei [$R_0$: \{$c_1^1$$\leftarrow$1; $c_1^1$$\leftarrow$1; $c_1^1$$\leftarrow$1\} , $R_1$: \{$\leftarrow$$c_1^1$\}, $R_2$: \{$\leftarrow$$c_1^1$\}]\\sowie Erkennung der
    potenziellen Kommunikationspartner} & Ja \\ \hline
  27 & \makecell[l]{Keine Probleme bei [$R_0$: \{$c_1^1$$\leftarrow$1; $c_1^1$$\leftarrow$1\} , $R_1$: \{$\leftarrow$$c_1^1$\}, $R_2$: \{$\leftarrow$$c_1^1$\}]} & Ja \\ \hline
  28 & \makecell[l]{Kein Kommunikationspartner wenn Receive in Fork bei ungebuffertem Channel\\$[$$R_0$: \{$c_1^0$ $\leftarrow$ 1; fork $R_1$\}, $R_1$: \{$\leftarrow$ $c_1^0$\}$]$} & Ja \\ \hline
  29 & \makecell[l]{Mögliche Kommunikationspartner wenn Receive in Fork bei gebuffertem Channel\\$[$$R_0$: \{$c_1^1$ $\leftarrow$ 1; fork $R_1$\}, $R_1$: \{$\leftarrow$ $c_1^1$\}$]$} & Ja \\ \hline
  30 & \makecell[l]{Deadlock bei Wahl eines bestimmten Select-Case\\$[$$R_0$: \{$\leftarrow$$c_3^0$\}, $R_1$: \{$c_1^0$ $\leftarrow$ 1\}, $R_2$: \{$c_2^0$ $\leftarrow$ 1\},\\$R_3$: \{select \{ case $\leftarrow$ $c_1^0$ $\Rightarrow$ \{$c_3^0$ $\leftarrow$ 1\}, case $\leftarrow$ $c_2^0$ $\Rightarrow$ \{$\leftarrow$ $c_3^0$\}\}\}$]$} & Ja \\ \hline
  31 & \makecell[l]{Deadlock bei Wahl eines bestimmten Select-Case\\$[$$R_0$: \{$\leftarrow$$c_3^0$\}, $R_1$: \{$c_1^0$ $\leftarrow$ 1\}\\$R_2$: \{select \{ case $\leftarrow$ $c_1^0$ $\Rightarrow$ \{$c_3^0$ $\leftarrow$ 1\}; default $\Rightarrow$ \{$\leftarrow$ $c_3^0$\}\}\}$]$} & Ja \\ \hline
  32 &Tatsächliches Send auf geschlossenen Channel & Ja \\ \hline
  33 & Potenzielles aber nicht tatsächliches Send auf geschlossenen Channel Channel & Ja \\ \hline
  34 & \makecell[l]{Kein Problem, wenn Channel erst nach letztem Send geschlossen werden\\kann [$R_0$: \{$c_1^0$ $\leftarrow$ 1; close($c_1^1$)\}, $R_1$: \{$\leftarrow$$c_1^1$\}]} & Ja \\ \hline
  35 & \makecell[l]{Kein Problem, wenn Channel erst nach letztem Send geschlossen werden\\kann [$R_0$: \{$\leftarrow$$c_1^0$; close($c_1^0$)\}, $R_1$: \{$c_1^1$$\leftarrow$1\}]} & FP \\ \hline
  36 & \makecell[l]{Korrekte Kommunikationspartner bei [$R_0$: \{$c_1^1$ $\leftarrow$ 1; $c_1^1$ $\leftarrow$ 1; $c_1^1$ $\leftarrow$ 1\},\\$R_1$: \{$\leftarrow$$x$; $\leftarrow$$x$\}] (letztes Send hat keinen Kommunikationspartner)} & Ja \\ \hline
  37 & \makecell[l]{Deadlock, da gleichzeitiges Send und Receive durch Mutex Lock verhindert\\wird [$R_0$: \{$m_1$.Lock; $\leftarrow$$c_1^0$; $m_1$.Unlock\}, $R_1$: \{$m_1$.Lock; $c_1^0$ $\leftarrow$ 1; $m_1$.Unlock\}]} & Ja \\ \hline
  38 & \makecell[l]{Kein Problem, da gleichzeitiges Send und Receive durch RWMutex R-Lock nicht\\verhindert wird\\$[$$R_0$: \{$m_1^r$.RLock; $\leftarrow$$c_1^0$; $m_1^r$.RUnlock\}, $R_1$: \{$m_1^r$.RLock; $c_1^0$ $\leftarrow$ 1; $m_1^r$.RUnlock\}$]$} & Ja \\ \hline
  39 & \makecell[l]{Kein potenzielles zyklisches Locking da Operationen durch Channel-Operation\\getrennt sind [$R_0$: \{$\leftarrow$$c_1^0$; $m_1$.Lock; $m_2$.Lock; $m_2$.Unock; $m_1$.Unlock;\},\\$R_1$: \{$m_2$.Lock; $m_1$.Lock; $m_1$.Unock; $m_2$.Unlock; $c_1^0$ $\leftarrow$ 1\}]} & FP \\ \hline
  40 & \makecell[l]{Tatsächlicher Deadlock, da Send durch Lock nicht erreicht werden kann\\$[$$R_0$: \{$m_1$.Lock; $c_1^0$$\leftarrow$1; $m_1$.Unlock\}, $R_1$: \{$m_1$.Lock; $\leftarrow$$c_1^0$; $m_1$.Unlock\}$]$} & Ja\\ \hline
  41 & \makecell[l]{Potenzieller aber nicht tatsächlicher Deadlock ($R_1$ vor $R_2$), da Send durch Lock\\nicht erreicht werden könnte ($R_2$ vor $R_1$)\\$[$$R_0$: \{$m_1$.Lock; $c_1^0$$\leftarrow$1; $m_1$.Unlock\}, $R_1$: \{$m_1$.Lock; $\leftarrow$$c_1^0$; $m_1$.Unlock\}$]$} & FP \\ \hline
  42 & \makecell[l]{Potenzielles zyklisches Locking bei Wahl eines Select-Cases\\$[$$R_0$: \{$m_1$.Lock; $m_2$.Lock; $m_2$.Unlock; $m_1$.Unlock\},\\$R_1$: \{select \{case $\leftarrow c_1^0 \Rightarrow$ \{$m_2$.Lock; $m_1$.Lock; $m_1$.Unlock; $m_2$.Unlock\}, default $\Rightarrow$ \{\}\};\\$R_2$: \{$c_1^0$ <- 1\}\}$]$} & Ja \\ \hline
  43 & \makecell[l]{Potenzielles zyklisches Locking bei Wahl eines DefaultSelect-Cases\\$[$$R_0$: \{$m_1$.Lock; $m_2$.Lock; $m_2$.Unlock; $m_1$.Unlock\},\\$R_1$: \{select \{case $\leftarrow c_1^0 \Rightarrow$ \{\}, default $\Rightarrow$ \{$m_2$.Lock; $m_1$.Lock; $m_1$.Unlock; $m_2$.Unlock\}\};\\$R_2$: \{$c_1^0$ <- 1\}\}$]$} & Ja \\ \hline
  \caption{Beschreibung der für die Auswertung betrachteten 
  Standartsituationen. In den Fällen in denen 
  Teile des Programmcodes angegeben sind sei $R_0$ die Main-Routine. $c_i^j$
  sei ein Channel mit Kapazität $j$, $m_i$ ein Mutex und $m_i^r$ ein RWMutex.
  Die Ergenbnisse und befinden sich in Tab.~\ref{App-Stand-Res}}
  \label{App-Stand-Des}
\end{longtable}


\newpage

\begin{longtable}[c]{|l|l|l|c|}
  \hline
  \textbf{Id} & \textbf{Type}  & \textbf{SubType}    & \multicolumn{1}{l|}{\textbf{Ok?}} \\ \hline
  \endfirsthead
  %
  \endhead
  %
  Cockroach 584      & Resource Deadlock      & Doppeltes Locking      & Ja                                   \\ \hline
  moby 17176      & Resource Deadlock      & Doppeltes Locking      & Ja                                   \\ \hline
  moby 36114      & Resource Deadlock      & Doppeltes Locking      & Ja                                   \\ \hline
  etcd 5509      & Resource Deadlock      & Doppeltes Locking      & Ja                                   \\ \hline
  etcd 6708      & Resource Deadlock      & Doppeltes Locking      & Ja                                   \\ \hline
  moby 7559      & Resource Deadlock      & Doppeltes Locking      & Ja                                   \\ \hline
  serving 4829      & Resource Deadlock      & Doppeltes Locking      & Ja                                   \\ \hline
  Cockroach 9935      & Resource Deadlock      & Doppeltes Locking      & Ja                                   \\ \hline
  moby 4951      & Resource Deadlock      & Zyklisches Locking     & Ja                                   \\ \hline
  Cockroach 7504      & Resource Deadlock      & Zyklisches Locking     & Ja                                   \\ \hline
  Cockroach 10214      & Resource Deadlock      & Zyklisches Locking     & Ja                                   \\ \hline
  hugo 3251      & Resource Deadlock      & Zyklisches Locking     & Ja                                   \\ \hline
  kubernetes 13135      & Resource Deadlock      & Zyklisches Locking     & Ja                                   \\ \hline
  Cockroach 6181      & Resource Deadlock      & RWR Deadlock           & Ja                                   \\ \hline
  kubernetes 58107      & Resource Deadlock      & RWR Deadlock           & FN                                 \\ \hline
  Cockroach 24808      & Communication Deadlock & Channel                & Ja                                   \\ \hline
  Cockroach 25456      & Communication Deadlock & Channel                & Ja                                   \\ \hline
  Cockroach 35073      & Communication Deadlock & Channel                & Ja                                   \\ \hline
  Cockroach 35931      & Communication Deadlock & Channel                & Ja                                   \\ \hline
  etcd 6857      & Communication Deadlock & Channel                & Ja                                   \\ \hline
  kubernetes 38669      & Communication Deadlock & Channel                & Ja                                   \\ \hline
  kubernetes 5316      & Communication Deadlock & Channel                & FN                                 \\ \hline
  kubernetes 70277      & Communication Deadlock & Channel                & Ja                                   \\ \hline
  moby 4395      & Communication Deadlock & Channel                & Ja                                   \\ \hline
  moby 29733      & Communication Deadlock & Konditionelle Variable & Ja                                   \\ \hline
  moby 30408      & Communication Deadlock & Konditionelle Variable & Ja                                   \\ \hline
  etcd 6873      & Mixed Deadlock         & Channel\&Mutex         & Ja                                   \\ \hline
  etcd 7902      & Mixed Deadlock         & Channel\&Mutex         & Ja                                   \\ \hline
  istio 16224      & Mixed Deadlock         & Channel\&Mutex         & Ja                                   \\ \hline
  kubernetes 10182      & Mixed Deadlock         & Channel\&Mutex         & Ja                                   \\ \hline
  kubernetes 1321      & Mixed Deadlock         & Channel\&Mutex         & FN                                 \\ \hline
  kubernetes 26980      & Mixed Deadlock         & Channel\&Mutex         & Ja                                   \\ \hline
  kubernetes 6632      & Mixed Deadlock         & Channel\&Mutex         & Ja                                   \\ \hline
  serving 2137      & Mixed Deadlock         & Channel\&Mutex         & Ja                                   \\ \hline
  \caption{Für Auswertung betrachteten 
  Programme aus Gobench~\cite{gobench} sowie 
  Information (Ok?) ob die Situation korrekt erkannt wurde. Ja bedeutet, dass 
  die Situation korrekt erkannt wurde, FP bezeichnet ein False-Positive
  und FN ein False-Negative. }
  \label{App-Goker}
  \end{longtable}


\chapter{Bespielprogramm}\label{Appendix-2}
Man betrachte das folgende Beispielprogramm.
  \lstinputlisting{code/show.txt}
Nach der Instrumentierung erhält man
\lstinputlisting{code/show_inst.txt}
sowie die neue Main-Datei
\lstinputlisting{code/show_main.txt}

Dieses Programm besitzt den folgenden Output:
\begin{verbatim}
Determine switch execution order
Start Program Analysis
Analyse Program:   0%   1,0
Analyse Program:  50%   1,1
Analyse Program: 100%

Finish Analysis

Found Problems:

Found while examine the following orders:   1,0  1,1
Potential Cyclic Mutex Locking:
Lock: /home/.../output/show/main.go:109
  Hs:
    /home/.../output/show/main.go:108
Lock: /home/.../output/show/main.go:100
  Hs:
    /home/.../output/show/main.go:99


Found while examine the following orders:   1,0
Possible Send to Closed Channel:
    Close: /home/.../output/show/main.go:48
    Send: /home/.../output/show/main.go:112

Found while examine the following orders:   1,1
No communication partner for receive at /home/.../output/show/main.go:103 
when running the following communication:
    /home/.../output/show/main.go:112 -> /home/.../output/show/main.go:65
    /home/.../output/show/main.go:39 -> /home/.../output/show/main.go:41

No communication partner for receive at /home/.../output/show/main.go:65 
when running the following communication:
    /home/.../output/show/main.go:112 -> /home/.../output/show/main.go:103
    /home/.../output/show/main.go:39 -> /home/.../output/show/main.go:41
\end{verbatim}
Die Pfade wurden dabei hier für eine bessere Lesbarkeit gekürzt.
Die Positionsangaben beziehen sich dabei immer auf die Position im 
instrumentieren Code.   
